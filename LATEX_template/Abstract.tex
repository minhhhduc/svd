\begin{abstract}
\text{}\indent Trong kỷ nguyên dữ liệu lớn, việc xử lý và phân tích các tập dữ liệu nhiều chiều đặt ra những thách thức to lớn về chi phí tính toán và lưu trữ. Giảm chiều dữ liệu (Dimensionality Reduction) là một bước tiền xử lý quan trọng nhằm giải quyết vấn đề này, trong đó Phân rã giá trị kỳ dị (Singular Value Decomposition - SVD) đóng vai trò nền tảng. Đồ án này tập trung nghiên cứu và triển khai thuật toán SVD song song nhằm tối ưu hóa hiệu năng tính toán trên các hệ thống đa lõi.

Cụ thể, chúng tôi đã xây dựng thuật toán \textit{Parallel Norm-Reducing Jacobi} để tính toán trị riêng và vector riêng, đồng thời song song hóa các bài toán con cốt lõi như nhân ma trận (Cannon, DNS), chuyển vị ma trận và sắp xếp dữ liệu sử dụng thư viện OpenMP. Hiệu năng của các thuật toán được đánh giá thực nghiệm trên nhiều kích thước dữ liệu khác nhau và so sánh với các thư viện chuẩn như LAPACK.

Kết quả thực nghiệm cho thấy phiên bản song song đạt hệ số tăng tốc (speedup) lên tới 4.6 lần so với phiên bản tuần tự trên các tập dữ liệu lớn. Bên cạnh đó, tính ứng dụng của hệ thống được minh chứng qua bài toán giảm chiều dữ liệu ảnh chữ số viết tay MNIST, giúp giảm không gian đặc trưng từ 784 xuống 50 chiều trong khi vẫn duy trì độ chính xác phân loại xấp xỉ 97\% với mô hình Multi-layer Perceptron. Kết quả này khẳng định tiềm năng và hiệu quả của việc áp dụng kỹ thuật tính toán song song trong các bài toán đại số tuyến tính và học máy.
\end{abstract}
